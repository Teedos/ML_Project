\documentclass{article}

% if you need to pass options to natbib, use, e.g.:
%     \PassOptionsToPackage{numbers, compress}{natbib}
% before loading neurips_2021

% ready for submission
\usepackage{neurips_2021}

% to compile a preprint version, e.g., for submission to arXiv, add add the
% [preprint] option:
%     \usepackage[preprint]{neurips_2021}

% to compile a camera-ready version, add the [final] option, e.g.:
%     \usepackage[final]{neurips_2021}

% to avoid loading the natbib package, add option nonatbib:
%    \usepackage[nonatbib]{neurips_2021}

\usepackage[utf8]{inputenc} % allow utf-8 input
\usepackage[T1]{fontenc}    % use 8-bit T1 fonts
\usepackage{hyperref}       % hyperlinks
\usepackage{url}            % simple URL typesetting
\usepackage{booktabs}       % professional-quality tables
\usepackage{amsfonts}       % blackboard math symbols
\usepackage{nicefrac}       % compact symbols for 1/2, etc.
\usepackage{microtype}      % microtypography
\usepackage{xcolor}         % colors

\title{Big data bowl 2022}

% The \author macro works with any number of authors. There are two commands
% used to separate the names and addresses of multiple authors: \And and \AND.
%
% Using \And between authors leaves it to LaTeX to determine where to break the
% lines. Using \AND forces a line break at that point. So, if LaTeX puts 3 of 4
% authors names on the first line, and the last on the second line, try using
% \AND instead of \And before the third author name.

\author{
  Massimo Hong\\
  Department of Computer Science\\
  Tsinghua University\\
  \texttt{hongcd21@mails.tsinghua.edu.cn} \\
  Student id: 2020280082}
  % examples of more authors
  % \And
  % Coauthor \\
  % Affiliation \\
  % Address \\
  % \texttt{email} \\
  % \AND
  % Coauthor \\
  % Affiliation \\
  % Address \\
  % \texttt{email} \\
  % \And
  % Coauthor \\
  % Affiliation \\
  % Address \\
  % \texttt{email} \\
  % \And
  % Coauthor \\
  % Affiliation \\
  % Address \\
  % \texttt{email} \\
%}

\begin{document}

\maketitle

\begin{abstract}
  NFL is one of the biggest and most prestigious sport league in the United States, with the highest average attendance of any sport in the world. \\
The kaggle competitions has a list of compulsory, but not limited, tasks to complete. In our project we will also gather and filter data from the seasons 2018-2020, analise the various statistics and feed the obtained parameters to the neural network in order to get the win probability of a team against a rival. In the end, we will give the same inputs to a logistic regression model and compare the respective outputs. Other than win probability, it could also be possible to predict whether a certain play or strategy will be successfull or not. \\
A neural network \textit{should} be more accurate than a logistic regression, that is essentially just a "subset" of the former. 
\end{abstract}
\section{Introduction to NFL}
The National Football League (NFL) is a professional American football league consisting of 32 teams, divided equally between the National Football Conference (NFC) and the American Football Conference (AFC). The NFL is one of the four major North American professional sports leagues, the highest professional level of American football in the world. \\
The NFL's eighteen-week regular season runs from early September to early January, with each team playing seventeen games and having one bye week. Following the conclusion of the regular season, seven teams from each conference (four division winners and three wild card teams) advance to the playoffs, a single-elimination tournament culminating in the Super Bowl, which is usually held on the first Sunday in February and is played between the champions of the NFC and AFC\cite{NFL_Wiki}.
\section{Dataset}
The dataset is divided into multiple csv files:
\begin{itemize}
\item game.csv, contains teams playing involved in each game.
\item plays.csv, contains  information regarding the plays of a game.
\item players.csv, contains information regarding the players.
\item tracking.csv, contains player tracking for each season.
\item PFFScoutingData.csv, contains play level scouting for each game.
\end{itemize}
\section{Data analysis}
We can use the gathered information to determine factors like:
\begin{itemize}
\item what play causes a team to win or lose the most points.
\item the most important attacking player on the team.
\item if the team has an advantage against the rival team based on the players, or the overall playstyle (e.g. difference in height and weight can mean a lot in this sport).
\item what could be the contribution of a certain player to the team.
\end{itemize}
All these factors can be used to better predict the outcome of a certain play, strategy, or the match itself.
\section{Required tasks of the competition}
\begin{itemize}
\item Create a new special teams metric.
\item Quantify special teams strategy. Special teams’ coaches are among the most creative and innovative in the league. Compare and contrast how each team game plans. Which strategies yield the best results? What are other strategies that could be adopted\cite{NFL_Kaggle}?
\item Rank special teams players. Each team employs a variety of players (including longsnappers, kickers, punters, and other utility special teams players). How do they stack up with respect to one another\cite{NFL_Kaggle}?
\end{itemize}

\section{Which neural network to use}
We will divide the data into training set and test set. The training set will be approximately 2/3 of the total data set.  \\
A possible choice for the neural network is the Probabilistic neural network (\textbf{PNN}), as these kind of networks usually generate an accurate target probability score.\\
Anothere option is to use a classification network to predict whehter a play was successfull or not.
\section{Regression Model}
It is more natural to use a logistic regression approach rather than a linear model, even though the output is not exactly binary. A few tweaks might be necessary in order to make it work correctly, such as using odds instead of probabilities, for an easier interpretation.
The advantages of a regression model is its easy implementation and ready to use methods offered by the python \textit{sklearn} library.
\section{Conclusion}
After getting the results from the neural network and the regression model, we can proceed to compare them. If everything is implemented correctly, the neural network should have a higher accuracy score than the regression model.

\bibliographystyle{unsrt}
\bibliography{bibliography.bib}
\end{document}